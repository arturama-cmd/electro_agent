\documentclass{article}
\usepackage[left=0.8cm,top=0.5cm,right=0.8cm,bottom=1.2cm]{geometry} 

\usepackage[spanish]{babel}
\usepackage{amsmath}
\usepackage{amssymb}
%\usepackage{enumitem}
\usepackage[american]{circuitikz}
\usepackage{enumerate}
\usepackage{tikz}
\usepackage{tikz-3dplot}
\usepackage{siunitx}
\begin{document}

\begin{enumerate}



\item[1.] \textcolor{red}{(\textbf{19 pts})} La figura muestra un circuito con cinco capacitores de capacitancias $C_1=C_2 =C$, $C_3 =C_4 = 2C$ y $C_5=5C$, donde $C$ es una constante con unidades de faradios ($\mathrm{F}$). Si $C=2{,}0\,\mu\mathrm{F}$ y $V=10\,\mathrm{V}$, determine:

\begin{minipage}{0.6\textwidth}
\begin{enumerate}
	\item La capacitancia equivalente. \textcolor{red}{(\textbf{10 pts})}

\item [$b)$] La carga total del circuito.\textcolor{red}{(\textbf{3 pts})}
\item [$c)$] El voltaje en $C_4$ y $C_5$.\textcolor{red}{(\textbf{6 pts})}

\end{enumerate}
\paragraph{Solución:}
\end{minipage}
\hfill
\begin{minipage}{0.4\textwidth}
\begin{center}
\begin{circuitikz}[scale=0.7]
    % Batería y punto P
    \draw (0,0) 
        to[battery,invert, l=$V$] (0,4) coordinate[label=above:] (p)
        to[short] (3,4);
    
    % Rama principal superior
    \draw (3,4)
        to[C, l=$C_1$] (3,2)
        to[short] (3,2)
        to[C, l=$C_2$] (3,0)
        to[short] (0,0)
       % to[C,l=$C_{3}$] (3,0)
       ;
    
    % Capacitores verticales
    \draw (3,4)
    to[short] (3,4)
    to[C, l=$C_3$] (6,4)
        to[C, l=$C_4$] (6,2)
        to[C,l=$C_5$] (6,0)
        to[short] (3,0)
        ;
    \draw (3,2)
        to[short] (6,2);
    
    % Título de la figura
   % \node[below] at (3.5,-0.5) {Figura 1};
\end{circuitikz}
\end{center}
\end{minipage}
\paragraph{Solución:}
\begin{enumerate}
    \item[(a)] Capacitancia equivalente

    Los capacitores $C_3$ y $C_4$ están en serie:
    \[
    C_{34} = \frac{C_3 C_4}{C_3 + C_4} = \frac{(2C)(2C)}{2C + 2C} = \frac{4C^2}{4C} = C \textcolor{red}{\quad (2\text{ pts})}
    \]
    
    Este resultado está en paralelo con $C_1$:
    \[
    C_{134} = C_{34} + C_1 = C + C = 2C \textcolor{red}{\quad (2\text{ pts})}
    \]
    
    Por otro lado, $C_2$ y $C_5$ están en paralelo:
    \[
    C_{25} = C_2 + C_5 = C + 5C = 6C \textcolor{red}{\quad (2\text{ pts})}
    \]
    
    Finalmente, $C_{134}$ y $C_{25}$ están en serie:
    \[
    C_{eq} = \frac{C_{134} \cdot C_{25}}{C_{134} + C_{25}} = \frac{2C \cdot 6C}{2C + 6C} = \frac{12C^2}{8C} = \frac{3C}{2}
    \]
    
    \[
    \boxed{C_{eq} = \frac{3}{2}C} \textcolor{red}{\quad (2\text{ pts})}
    \]
Entonces, capacitancia equivalente:
        \[
        C_{eq} = \frac{3}{2}C = \frac{3}{2} \cdot 2\,\mu F = 3\,\mu F\textcolor{red}{\quad (2\text{ pts})}
        \]
    \item[(b)] La carga total del circuito.\textcolor{red}{(\textbf{3 pts})}.\\


    \begin{itemize}

        
        \item Carga total:
        \[
        Q_{total} = C_{eq} \cdot V = 3\,\mu F \cdot 10\,V = 30\,\mu C
        \]
         \[\therefore ~
        \boxed{Q_{total} = 30\,\mu C}\textcolor{red}{\quad (3\text{ pts})}
        \]
        
        \end{itemize}
       \item[(c)] El voltaje en $C_4$ y $C_5$. \textcolor{red}{(\textbf{6 pts})}\\
       
            \begin{itemize}
            
        \item Voltajes en la rama superior ($C_{134} = 4\,\mu F$) y la rama inferior ($C_{25} = 12\,\mu F$), como están en serie:
        
        \[
        Q_{134} = Q_{25} = Q_{total} = 30\,\mu C
        \]
        
        \[
        V_{134} = \frac{Q}{C_{134}} = \frac{30\,\mu C}{4\,\mu F} = 7.5\,V
        \]
        
        \[
        V_{25} = \frac{Q}{C_{25}} = \frac{30\,\mu C}{12\,\mu F} = 2.5\,V
        \]
        
        \[\therefore~
        \boxed{V_{C5} =  2.5\,V}\textcolor{red}{\quad (3\text{ pts})}
        \]
        
        \item El voltaje en $C_4$. $C_1$ y $C_{34}$ en paralelo comparten el mismo voltaje:
        
        \[
        V_{C1} = V_{C34} = 7.5\,V
        \]
        
           Para $C_{34}$:
        \[
        Q_{34} = C_{34} \cdot V_{C34} = 2\,\mu F \cdot 7.5\,V = 15\,\mu C
        \]
        
        Como $C_3$ y $C_4$ están en serie, tienen la misma carga:
        
        \[
        Q_{C3} = Q_{C4} = Q_{34} = 15\,\mu C
        \]
        
      
        \[\therefore~
        \boxed{V_{C4} = \frac{Q_{C4}}{C_4} = \frac{15\,\mu C}{4\,\mu F} = 3.75\,V} \textcolor{red}{\quad (3\text{ pts})}
        \]

         
    \end{itemize}

   

\end{enumerate}

\newpage
\item[1.] \textcolor{red}{(\textbf{19 pts})} La figura muestra un circuito con cinco capacitores de capacitancias $C_1=C_2 =C$, $C_3 =C_4 = 2C$ y $C_5=5C$, donde $C$ es una constante con unidades de faradios ($\mathrm{F}$). Si $C=2{,}0\,\mu\mathrm{F}$ y $V=11\,\mathrm{V}$, determine:

\begin{minipage}{0.5\textwidth}
\begin{enumerate}
	\item La capacitancia equivalente. \textcolor{red}{(\textbf{8 pts})}
%\end{enumerate}
	
%Si $C=2{,}0\,\mu\mathrm{F}$ y $V=11\,\mathrm{V}$, determine:
%\begin{enumerate}
\item La carga total del circuito.\textcolor{red}{(\textbf{3 pts})}
\item  El voltaje y la carga en $C_4$ y $C_5$.\textcolor{red}{(\textbf{8 pts})}

\end{enumerate}
\paragraph{Solución:}
\end{minipage}
\hfill
\begin{minipage}{0.5\textwidth}
\begin{center}
\begin{circuitikz}[scale=0.7]
    % Batería y punto P
    \draw (0,0) 
        to[battery,invert, l=$V$] (0,4) coordinate[label=above:] (p)
        to[short] (3,4);
    
    % Rama principal superior
    \draw (3,4)
        to[C, l=$C_1$] (3,2)
        to[short] (3,2)
        to[C, l=$C_2$] (3,0)
        to[short] (0,0)
       % to[C,l=$C_{3}$] (3,0)
       ;
    
    % Capacitores verticales
    \draw (3,4)
    to[short] (6,4)
        to[C, l=$C_3$] (6,2)
                to[C,l=$C_5$] (6,0)
        to[short] (3,0)
        ;
    \draw (3,2)
        to[short] (6,2);
    \draw (6,4) to[short] (9,4)
    to[C, l=$C_4$] (9,2)
    to[short] (6,2)
    ;
    
    % Título de la figura
   % \node[below] at (3.5,-0.5) {Figura 1};
\end{circuitikz}
\end{center}
\end{minipage}
\paragraph{Solución:}

\begin{enumerate}
    \item[(a)] Capacitancia equivalente

    Los capacitores  $C_1$, $C_3$ y $C_4$ están en paralelo:

    \[
    C_{134} = C_3+C_4 + C_1 = 2C+2C + C = 5C \textcolor{red}{\quad (2\text{ pts})}
    \]
    
    Por otro lado, $C_2$ y $C_5$ están en paralelo:
    \[
    C_{25} = C_2 + C_5 = C + 5C = 6C \textcolor{red}{\quad (3\text{ pts})}
    \]
    
    Finalmente, $C_{134}$ y $C_{25}$ están en serie:
    \[
    C_{eq} = \frac{C_{134} \cdot C_{25}}{C_{134} + C_{25}} = \frac{5C \cdot 6C}{5C + 6C} = \frac{30C^2}{11C} = \frac{30C}{11}
    \]
    
    \[
    \boxed{C_{eq} = 5.45\,\mu F} \textcolor{red}{\quad (3\text{ pts})}
    \]

    \item[(b)] La carga total del circuito.\textcolor{red}{(\textbf{3 pts})}.\\


    \begin{itemize}
        \item Carga total:
 \[
    Q_{total} = C_{eq} \cdot V = \frac{30}{11}C \cdot 11V = 30C = 60\,\mu C
    \]
         \[\therefore ~
        \boxed{Q_{total} = 60\,\mu C}\textcolor{red}{\quad (3\text{ pts})}
        \]
        
        \end{itemize}
       \item[(c)] El voltaje y la carga en $C_4$ y $C_5$.\textcolor{red}{(\textbf{8 pts})}\\
       
            \begin{itemize}
            
        \item Voltajes en la rama superior ($C_{134} = 10\,\mu F$) y la rama inferior ($C_{25} = 12\,\mu F$), como están en serie:
        
        \[
        Q_{134} = Q_{25} = Q_{total} = 60\,\mu C
        \]
        
        \[
        V_{134} = \frac{Q}{C_{134}} = \frac{60\,\mu C}{10\,\mu F} = 6.0\,V
        \]
        
        \[
        V_{25} = \frac{Q}{C_{25}} = \frac{60\,\mu C}{12\,\mu F} = 5.0\,V
        \]
        
       
        
        \item Como los capacitores $C_1$ $C_3$ y $C_{4}$ están paralelo, comparten el mismo voltaje:
        
         \[\therefore~
        \boxed{V_{C4} =6.0\,V} \textcolor{red}{\quad (2\text{ pts})}
        \]
        \item $C_2$ y $C_5$ están en paralelo:
        
         \[\therefore~
        \boxed{V_{C5} =  5.0\,V}\textcolor{red}{\quad (2\text{ pts})}
        \]
               \item La carga en $C_4$ y $C_5$:
        
         \[
        \boxed{Q_{C4} = C_{4}V_{4}=2\cdot 2\,\mu F\cdot 6,0\,V=24\,\mu\text C}\textcolor{red}{\quad (2\text{ pts})}
        \]
       
  \[
        \boxed{Q_{C5} = C_{5}V_{5}=5\cdot 2\,\mu F\cdot 5,0\,V=50\,\mu\text C}\textcolor{red}{\quad (2\text{ pts})}
        \]
         
    \end{itemize}

   

\end{enumerate}
\end{enumerate}
\end{document}
