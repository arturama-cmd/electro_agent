\documentclass[12pt]{article}
\usepackage[spanish]{babel}
\usepackage{amsmath}
\usepackage{siunitx}
\usepackage{enumitem}
\usepackage{geometry}
\usepackage{wrapfig}
\usepackage[american]{circuitikz}
\usepackage{tikz}
\usetikzlibrary{decorations.markings}  % Para flechas en arcos

\geometry{margin=2.5cm}

\begin{document}


(20 pts) La figura muestra un circuito que combina resistencias y baterías en distintas configuraciones.  
Considere los sentidos de las corrientes $\displaystyle I_{1}$, $\displaystyle I_{2}$ e $\displaystyle I_{3}$, así como el recorrido de malla indicado, y responda:\\
\begin{wrapfigure}{r}{0.5\textwidth}
  \centering
  \begin{circuitikz}[scale=0.8, every node/.style={font=\small}]
  \draw (-1,1.5) to[battery1,l_=12\,\mathrm{V}] (-1,-1);
  \draw (-1,1.5) -- (0,1.5);
  \draw (0,1.5)  to[R,l=4~$\Omega$] (2,1.5);
  \draw (2,1.5) to[short, i>^={$I_1$}] (4,1.5);
  \draw (4,1.5) to[short] (4,1)
               to[R,l_=6~$\Omega$, i<^={$I_2$}] (4,-3);
  \draw (4,1.5) to[battery1,l=3\,\mathrm{V}, i>^={$I_3$}] (6,1.5);
  \draw (6,1.5) -- (9,1.5);
  \draw (6,1.5)  to[short] (6,1)    to[R,l=2~$\Omega$] (6,-3);
  \draw (7.5,1.5) to[short] (7.5,1) to[R,l=3~$\Omega$] (7.5,-3);
  \draw (9,1.5)  to[short] (9,1)    to[R,l=6~$\Omega$] (9,-3);
  \draw (6,-3) -- (9,-3);
  \draw (2,-3) to[battery1,l_=6\,\mathrm{V}] (4,-3);
  \draw (4,-3) to[battery1,l_=14\,\mathrm{V}] (6,-3);
  \draw (-1,-1) to[R,l=18~$\Omega$] (2,-1);
  \draw (-1,-1) to[R,l=9~$\Omega$]  (2,-3);
  \draw (-1,-1) -- (-1,-3)
               to[R,l_=3~$\Omega$] (2,-3);
  \draw (2,-1) -- (2,-3);

  % Malla M1
  \coordinate (C1) at (2,0.25);
  \def\rA{0.6}
  \draw[blue, dashed, very thick,
        postaction={decorate},
        decoration={markings,
          mark=at position 1 with {\arrow[blue, scale=1.2]{>}}}]
    (C1)+(90:\rA) arc[start angle=90, end angle=360, radius=\rA];
  \node[blue] at (C1) {\small $M_1$};

  % Malla M2
  \coordinate (C2) at (5,-0.5);
  \def\rB{0.5}
  \draw[blue, dashed, very thick,
        postaction={decorate},
        decoration={markings,
          mark=at position 1 with {\arrow[blue, scale=1.2]{>}}}]
    (C2)+(90:\rB) arc[start angle=90, end angle=360, radius=\rB];
  \node[blue] at (C2) {\small $M_2$};


  \end{circuitikz}

\end{wrapfigure}



\begin{enumerate}[label=\textbf{\alph*)}, leftmargin=*, nosep]
  \item Simplifique el circuito todo lo posible obteniendo las resistencias equivalentes y dibuje el esquema reducido resultante (5 pts).
  \item Para el circuito reducido, escriba las ecuaciones de las leyes de Kirchhoff (corriente y voltaje) correspondientes a cada malla y a cada nodo relevante (3 pts).
  \item Resuelva el sistema obtenido para determinar los valores numéricos de las corrientes $\displaystyle I_{1}$, $\displaystyle I_{2}$ e $\displaystyle I_{3}$ (8 pts).
  \item Calcule la potencia disipada en cada resistencia del circuito reducido utilizando los resultados de la parte anterior (4 pts).
\end{enumerate}

\section*{Solución}

a)  El circuito se puede reducir con 2 resistencias equivalentes:

    
\begin{wrapfigure}{r}{0.5\textwidth}
  \centering
  \begin{circuitikz}[scale=0.8, every node/.style={font=\small}]
  \draw (-1,1.5) to[battery1,l_=12\,\mathrm{V}] (-1,-1);
  \draw (-1,1.5) -- (0,1.5);
  \draw (0,1.5)  to[R,l=4~$\Omega$] (2,1.5);
  \draw (2,1.5) to[short, i>^={$I_1$}] (4,1.5);
  \draw (4,1.5) to[short] (4,1)
               to[R,l_=6~$\Omega$, i<^={$I_2$}] (4,-3);
  \draw (4,1.5) to[battery1,l=3\,\mathrm{V}, i>^={$I_3$}] (6,1.5);
    \draw (6,1.5)  to[short] (6,1)    to[R,l=$R_{eq2}$] (6,-3);
    \draw (2,-3) to[battery1,l_=6\,\mathrm{V}] (4,-3);
  \draw (4,-3) to[battery1,l_=14\,\mathrm{V}] (6,-3);
  \draw (-1,-1) -- (-1,-3)
               to[R,l_=$R_{eq1}$] (2,-3);
  

  % Malla M1
  \coordinate (C1) at (1,-0.5);
  \def\rA{0.6}
  \draw[blue, dashed, very thick,
        postaction={decorate},
        decoration={markings,
          mark=at position 1 with {\arrow[blue, scale=1.2]{>}}}]
    (C1)+(90:\rA) arc[start angle=90, end angle=360, radius=\rA];
  \node[blue] at (C1) {\small $M_1$};

  % Malla M2
  \coordinate (C2) at (5,-0.5);
  \def\rB{0.5}
  \draw[blue, dashed, very thick,
        postaction={decorate},
        decoration={markings,
          mark=at position 1 with {\arrow[blue, scale=1.2]{>}}}]
    (C2)+(90:\rB) arc[start angle=90, end angle=360, radius=\rB];
  \node[blue] at (C2) {\small $M_2$};
  \end{circuitikz}

\end{wrapfigure}



\begin{eqnarray}
    \frac{1}{R_{eq1}}&=&\frac{1}{18}+\frac{1}{9}+\frac{1}{3} \nonumber\\
    &=&\frac{1+2+6}{18}\nonumber \\
    &=&\frac{9}{18} \quad\ /()^{-1} \nonumber \\
    R_{eq1}&=& 2 \Omega \nonumber
\end{eqnarray}

\begin{eqnarray}
    \frac{1}{R_{eq2}}&=&\frac{1}{2}+\frac{1}{3}+\frac{1}{6} \nonumber\\
    &=&\frac{3+2+1}{6}\nonumber \\
    &=&\frac{6}{6} \quad\ /()^{-1} \nonumber \\
    R_{eq2}&=& 1 \Omega \nonumber
\end{eqnarray}

\clearpage

b) Al aplicar las leyes de  Kirchhoff sobre el circuito reducido se obtiene:


\begin{eqnarray}
Nodo)\quad\quad\ &&I_1+I_2=I_3 \\
M1)\quad\quad\ &&4I_1-12+2I_1-6-6I_2=0 \\
M2)\quad\quad\ &&1\cdot I_3+3+6I_2-14=0
\end{eqnarray}

c)Utilizando las ecuaciones anteriores es posible determinar los valores de las corrientes solicitadas. Para eso iniciamos trabajando (2)

\begin{eqnarray}
4I_1-12+2I_1-6-6I_2&=&0 \nonumber\\
6I_1-18-6I_2&=&0 \quad\ /\frac{1}{6} \nonumber \\
I_1-3-I_2&=&0\nonumber\\
I_1&=&3+I_2
\end{eqnarray}

Reemplazando (4) en (1)

\begin{eqnarray}
3+I_2+I_2&=&I_3\nonumber\\
3+2I_2&=&I_3
\end{eqnarray}
Reemplazando (5) en (3)
\begin{eqnarray}
I_3+6I_2-11&=&I_3\nonumber\\
3+2I_2+6I_2-11&=&I_3\nonumber\\
8I_2&=&8\nonumber\\
I_2&=&1 A
\end{eqnarray}
Reemplazando $I_2$ en (4)

\begin{eqnarray}
I_1&=&3+1 \nonumber\\
I_1&=&4 A
\end{eqnarray}
Reemplazando $I_2$,$I_1$ en (1)

\begin{eqnarray}
I_1+I_2&=&I_3 \nonumber\\
I_3&=&4+1 \nonumber\\
I_3&=&5 A
\end{eqnarray}

d) Para determinar la potencia disipada por cada resistencia cuando el circuito esta reducido. Utilizamos:


\begin{eqnarray}
P=I^2R \nonumber\\
\end{eqnarray}

Así para las 4 resistencias que conforman el circuito reducido se obtiene:

\begin{eqnarray}
P_R{_{eq1}}=I_1^2\cdot R_{eq1}= (4A)^2\cdot 2\Omega =  32W     \nonumber\\
P_R{_{4\Omega}}=I_1^2\cdot R_{4\Omega}= (4A)^2 \cdot 4\Omega= 64W  \nonumber\\
P_R{_{6\Omega}}=I_2^2 \cdot R_{6\Omega}= (1A)^2\cdot  6\Omega=  6W  \nonumber\\
P_R{_{eq2}}=I_3^2 \cdot R_{eq2}=(5A)^2 \cdot 1\Omega=   25W        \nonumber
\end{eqnarray}






\newpage


(20 pts) La figura muestra un circuito que combina resistencias y baterías en distintas configuraciones.  
Considere los sentidos de las corrientes $\displaystyle I_{1}$, $\displaystyle I_{2}$ e $\displaystyle I_{3}$, así como el recorrido de malla indicado, y responda:\\
\begin{wrapfigure}{r}{0.5\textwidth}
  \centering
\begin{circuitikz}[every node/.style={font=\small}, scale=1]

\draw (0,4) to[battery1,l_=13\,\mathrm{V}] (0,3);
\draw (0,4) -- (0,5);
\draw (0,3) -- (0,1);

\draw (0,5) to[R,l=3~$\Omega$] (2,5);     
\draw (2,4) -- (2,5);                     
\draw (0,4) to[R,l_=6~$\Omega$] (2,4);      
\draw (0,3) to[R,l_=8~$\Omega$] (2,3);     
\draw (0,1) to[R,l_=12~$\Omega$ ] (2,1);      
\draw (2,3) to[R,l_=4~$\Omega$] (2,1);       

\draw (2,1) to[battery1,l_=5\,\mathrm{V},invert] (4,1);

\draw (4,5) to[R,l=5~$\Omega$] (4,1);
\draw (2,5) -- (4,5) to[R,l=7~$\Omega$] (6,5);
\draw (6,5) to[R,l=8~$\Omega$] (6,1);
\draw (4,1) to[battery1,l=15\,\mathrm{V}] (6,1);

% Corrientes
\draw (5,1)  to[short,i>^={$I_3$}] (4,1);  
\draw (4,2)  to[short,i>^={$I_1$}] (4,1);  
\draw (4,1)  to[short,i>^={$I_2$}] (3.3,1); 

% Malla M1
\coordinate (C1) at (3,3.2);
\def\rA{0.5}
\draw[blue, dashed, very thick,
  postaction={decorate},
  decoration={markings,
    mark=at position 1 with {\arrow[blue, scale=1.2]{>}}}]
  (C1)+(90:\rA) arc[start angle=90, end angle=260, radius=\rA];
\node[blue] at (3,3.2) {\small $M_1$};

% Malla M2
\coordinate (C2) at (4.6,3.5);
\def\rB{0.5}
\draw[blue, dashed, very thick,
  postaction={decorate},
  decoration={markings,
    mark=at position 1 with {\arrow[blue, scale=1.2]{>}}}]
  (C2)+(90:\rB) arc[start angle=-180, end angle=-360, radius=\rB];
\node[blue] at (5.1,4.1) {\small $M_2$};

\end{circuitikz}
\end{wrapfigure}



\begin{enumerate}[label=\textbf{\alph*)}, leftmargin=*, nosep]
  \item Simplifique el circuito todo lo posible obteniendo las resistencias equivalentes y dibuje el esquema reducido resultante (5 pts).
  \item Para el circuito reducido, escriba las ecuaciones de las leyes de Kirchhoff (corriente y voltaje) correspondientes a cada malla y a cada nodo relevante (3 pts).
  \item Resuelva el sistema obtenido para determinar los valores numéricos de las corrientes $\displaystyle I_{1}$, $\displaystyle I_{2}$ e $\displaystyle I_{3}$ (8 pts).
  \item Calcule la potencia disipada en cada resistencia del circuito reducido utilizando los resultados de la parte anterior (4 pts).
\end{enumerate}



\section*{Solución}

a) El circuito se puede reducir utilizando 4 resistencias equivalentes, de las cuales 3 quedaran en el resultado final. Se observa en la figura que las resistencias ubicadas en la esquina superior izquierda del circuito se encuentran en paralelo. Así:

\begin{wrapfigure}{r}{0.5\textwidth}
  \centering
\begin{circuitikz}[every node/.style={font=\small}, scale=1]

\draw (0,4) to[battery1,l_=13\,\mathrm{V}] (0,3);
\draw (0,4) -- (0,5);
\draw (0,3) -- (0,1);

\draw (0,5) to[R,l=$R_{eq1}$] (2,5);     

\draw (0,1) to[R,l_=$R_{eq3}$ ] (2,1);      


\draw (2,1) to[battery1,l_=5\,\mathrm{V},invert] (4,1);

\draw (4,5) to[R,l=5~$\Omega$] (4,1);
\draw (2,5) -- (4,5) to[R,l=$R_{eq4}$] (6,5);
\draw (6,5) to (6,1);
\draw (4,1) to[battery1,l=15\,\mathrm{V}] (6,1);

% Corrientes
\draw (5,1)  to[short,i>^={$I_3$}] (4,1);  
\draw (4,2)  to[short,i>^={$I_1$}] (4,1);  
\draw (4,1)  to[short,i>^={$I_2$}] (3.3,1); 

% Malla M1
\coordinate (C1) at (2,3.2);
\def\rA{0.5}
\draw[blue, dashed, very thick,
  postaction={decorate},
  decoration={markings,
    mark=at position 1 with {\arrow[blue, scale=1.2]{>}}}]
  (C1)+(90:\rA) arc[start angle=90, end angle=260, radius=\rA];
\node[blue] at (2,3.2) {\small $M_1$};

% Malla M2
\coordinate (C2) at (4.6,3.5);
\def\rB{0.5}
\draw[blue, dashed, very thick,
  postaction={decorate},
  decoration={markings,
    mark=at position 1 with {\arrow[blue, scale=1.2]{>}}}]
  (C2)+(90:\rB) arc[start angle=-180, end angle=-360, radius=\rB];
\node[blue] at (5.1,4.1) {\small $M_2$};

\end{circuitikz}
\end{wrapfigure}



\begin{eqnarray}
    \frac{1}{R_{eq1}}&=&\frac{1}{3}+\frac{1}{6}\nonumber\\
    &=&\frac{2+1}{3}\quad\ /()^{-1} \nonumber \\
     R_{eq1}&=& 1 \Omega \nonumber
\end{eqnarray}

Luego en la esquina inferior izquierda se observan 2 resistencias en serie ($8\Omega$ y $4\Omega$), reduciendo estas se obtiene:

\begin{eqnarray}
    {R_{eq2}}&=&8+4\nonumber\\
    R_{eq2}&=& 12 \Omega \nonumber
\end{eqnarray}


La resistencia $R_{eq2}$ queda en paralelo con la de valor $12 \Omega$, al reducir ambas se obtiene:


\begin{eqnarray}
    \frac{1}{R_{eq3}}&=&\frac{1}{12}+\frac{1}{12}=\frac{1}{6}\quad\ /()^{-1} \nonumber \\
    R_{eq3}&=& 6 \Omega \nonumber
\end{eqnarray}

\clearpage

Finalmente en la esquina superior derecha del circuito se observan que las resistencia de $7\Omega$ y $8\Omega$ se encuentran en serie, reduciendolas se obtiene:

\begin{eqnarray}
    {R_{eq4}}&=&7+8\nonumber\\
    R_{eq4}&=& 15 \Omega \nonumber
\end{eqnarray}



\setcounter{equation}{0}





b) Al aplicar las leyes de  Kirchhoff sobre el circuito reducido se obtiene:


\begin{eqnarray}
Nodo)\quad\quad\ &&I_1+I_3=I_2 \\
M1)\quad\quad\ &&5+5I_1+2I_2-13+6I_2=0 \\
M2)\quad\quad\ &&15+5I_1-15I_3=0
\end{eqnarray}

c) Utilizando las ecuaciones anteriores es posible determinar los valores de las corrientes solicitadas. Para eso iniciamos trabajando (3)

\begin{eqnarray}
    15+5I_1-15I_3&=&0 \quad\ /\cdot \frac{1}{5} \nonumber \\
    3+I_1-3I_3&=&0 \nonumber \\
    I_3=\frac{3+I_1}{3}
\end{eqnarray}

Reemplazando (4) en (1)

\begin{eqnarray}
    I_1+I_3&=&I_2 \nonumber \\
I_1+\frac{3+I_1}{3}&=& I_2 \nonumber \\
3I_1+3+I_1&=& 3 I_2 \nonumber \\
3+4I_1&=& 3 I_2 \nonumber \\
I_2&=& \frac{3+4I_1}{3}
\end{eqnarray}

Reemplazando (5) en (2)

\begin{eqnarray}
5+5I_1+2I_2-13+6I_2&=&0 \nonumber \\    
5I_1+8I_2-8&=&0 \nonumber \\   
5I_1+8( \frac{3+4I_1}{3})-8&=&0 /\cdot 3 \nonumber \\
15I_1+8(3+4I_1)-24&=&0\nonumber \\
15I_1+24+32I_1-24&=&0\nonumber \\
I_1&=&0 A
\end{eqnarray}
Reemplazando $I_1$ en (5)

\begin{eqnarray}
    I_2&=& \frac{3+4\cdot 0}{3} \nonumber \\
    I_2&=& 1 A
\end{eqnarray}

Reemplazando  $I_1$, $I_2$ en 1 se obtiene:

\begin{eqnarray}
    I_1+I_3&=&I_2 \nonumber \\
    0+I_3&=&1 A \nonumber \\
    I_3&=& 1 A
\end{eqnarray}


d) Para determinar la potencia disipada por cada resistencia cuando el circuito esta reducido. Utilizamos:


\begin{eqnarray}
P=I^2R \nonumber\\
\end{eqnarray}

Así para las 4 resistencias que conforman el circuito reducido se obtiene:

\begin{eqnarray}
P_R{_{eq1}}=I_2^2\cdot R_{eq1}= (1A)^2\cdot 1\Omega =  1W     \nonumber\\
P_R{_{eq3}}=I_2^2\cdot R_{eq3}= (1A)^2 \cdot 6\Omega= 6W  \nonumber\\
P_R{_{eq4}}=I_3^2\cdot R_{eq4}= (1A)^2 \cdot 15\Omega= 15W  \nonumber\\
P_R{_{5\Omega}}=I_1^2 \cdot R_{5\Omega}= (0)^2\cdot  6\Omega=  0W  \nonumber\\
\end{eqnarray}






\end{document}
