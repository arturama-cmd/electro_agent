% Sections won't have numbers
% Make the page area as large as possible
% Set the first level numbering to arabic and the second level
% to alphabetic. The remaining two levels are arabic
%Change to \pagestyle{empty} to suppress numbers on following pages
%\pagestyle{empty}
% Do not reset outermost enumerate counter


\documentclass{article}
%%%%%%%%%%%%%%%%%%%%%%%%%%%%%%%%%%%%%%%%%%%%%%%%%%%%%%%%%%%%%%%%%%%%%%%%%%%%%%%%%%%%%%%%%%%%%%%%%%%%%%%%%%%%%%%%%%%%%%%%%%%%%%%%%%%%%%%%%%%%%%%%%%%%%%%%%%%%%%%%%%%%%%%%%%%%%%%%%%%%%%%%%%%%%%%%%%%%%%%%%%%%%%%%%%%%%%%%%%%%%%%%%%%%%%%%%%%%%%%%%%%%%%%%%%%%
\usepackage{amsmath}
\usepackage{sw20exm2}
\usepackage[compat2]{geometry}

\setcounter{MaxMatrixCols}{10}
%TCIDATA{OutputFilter=LATEX.DLL}
%TCIDATA{Version=5.50.0.2953}
%TCIDATA{<META NAME="SaveForMode" CONTENT="1">}
%TCIDATA{BibliographyScheme=Manual}
%TCIDATA{Created=Sunday, August 26, 2012 12:26:36}
%TCIDATA{LastRevised=Tuesday, August 23, 2022 09:40:04}
%TCIDATA{<META NAME="ViewSettings" CONTENT="0">}
%TCIDATA{<META NAME="GraphicsSave" CONTENT="32">}
%TCIDATA{<META NAME="DocumentShell" CONTENT="Exams and Syllabi\SW\SW Exam #1 - 8.5x14 Page">}
%TCIDATA{CSTFile=Exam.cst}

\setlength{\topmargin}{-0.75in}
\setlength{\textheight}{12.25in}
\setlength{\oddsidemargin}{0.0in}
\setlength{\evensidemargin}{0.0in}
\setlength{\textwidth}{6.5in}
\def\labelenumi{\arabic{enumi}.}
\def\theenumi{\arabic{enumi}}
\def\labelenumii{(\alph{enumii})}
\def\theenumii{\alph{enumii}}
\def\p@enumii{\theenumi.}
\def\labelenumiii{\arabic{enumiii}.}
\def\theenumiii{\arabic{enumiii}}
\def\p@enumiii{(\theenumi)(\theenumii)}
\def\labelenumiv{\arabic{enumiv}.}
\def\theenumiv{\arabic{enumiv}}
\def\p@enumiv{\p@enumiii.\theenumiii}
\pagestyle{plain}
\setcounter{secnumdepth}{0}
\newtheorem{theorem}{Theorem}
\newtheorem{acknowledgement}[theorem]{Acknowledgement}
\newtheorem{algorithm}[theorem]{Algorithm}
\newtheorem{axiom}[theorem]{Axiom}
\newtheorem{case}[theorem]{Case}
\newtheorem{claim}[theorem]{Claim}
\newtheorem{conclusion}[theorem]{Conclusion}
\newtheorem{condition}[theorem]{Condition}
\newtheorem{conjecture}[theorem]{Conjecture}
\newtheorem{corollary}[theorem]{Corollary}
\newtheorem{criterion}[theorem]{Criterion}
\newtheorem{definition}[theorem]{Definition}
\newtheorem{example}[theorem]{Example}
\newtheorem{exercise}[theorem]{Exercise}
\newtheorem{lemma}[theorem]{Lemma}
\newtheorem{notation}[theorem]{Notation}
\newtheorem{problem}[theorem]{Problem}
\newtheorem{proposition}[theorem]{Proposition}
\newtheorem{remark}[theorem]{Remark}
\newtheorem{solution}[theorem]{Solution}
\newtheorem{summary}[theorem]{Summary}
\newenvironment{proof}[1][Proof]{\noindent\textbf{#1.} }{\ \rule{0.5em}{0.5em}}
\input{tcilatex}
\begin{document}


\begin{center}
\begin{equation*}
\FRAME{itbpF}{0.9392in}{0.6763in}{0in}{}{}{Figure}{\special{language
"Scientific Word";type "GRAPHIC";maintain-aspect-ratio TRUE;display
"USEDEF";valid_file "T";width 0.9392in;height 0.6763in;depth
0in;original-width 1.3552in;original-height 0.9669in;cropleft "0";croptop
"1";cropright "1";cropbottom "0";tempfilename
'RH2KPY00.bmp';tempfile-properties "XPR";}}
\end{equation*}%
\textbf{Electromagnetismo (2022-1)}

\textbf{Tarea 4}\emph{: Campo Magn\'{e}tico e Inducci\'{o}n electromagn\'{e}%
tica}

\textit{Profesor: Arturo Fern\'{a}ndez P\'{e}rez}

\textbf{Plazo m\'{a}ximo de entrega: Mi\'{e}rcoles 31 de Agosto de 2022 a
las 23:59 hrs.}
\end{center}

\begin{enumerate}
\item Tres conductores rectos y muy extensos, conducen corrientes $I_{1}=2,6$
[A]$,I_{2}=5,1$ [A] e $I_{3}=3,2$ [A], en las direcciones que indica la Fig.
1. En un instante de tiempo, una carga puntual $q$ = $5,8$ [$mC$] se mueve
con una rapidez $v=50,0$ $[m/s]$ en la direcci\'{o}n indicada en la Fig. 1.
Considerando que $a=5,4$ [cm], $b=2,8$ [cm] y $c=7,3$ [cm], determine:
(Utilice el sistema de referencia indicado en la Fig. 1)

\begin{itemize}
\item El campo magn\'{e}tico total $\vec{B}$ en la posici\'{o}n de la carga
puntual $q.$

\item La fuerza magn\'{e}tica neta $\vec{F}_{m}$ ejercida sobre la carga
puntual $q.$

\item La fuerza magn\'{e}tica $\vec{F}_{m}$ ejercida sobre una secci\'{o}n
de 2 [m] de longitud del lazo de corriente $I_{2}$, ejercida por la
corriente $I_{1}$.%
\begin{equation*}
\FRAME{itbpFU}{3.7974in}{3.7663in}{0in}{\Qcb{Figura 1}}{}{Figure}{\special%
{language "Scientific Word";type "GRAPHIC";maintain-aspect-ratio
TRUE;display "USEDEF";valid_file "T";width 3.7974in;height 3.7663in;depth
0in;original-width 3.7498in;original-height 3.7187in;cropleft "0";croptop
"1";cropright "1";cropbottom "0";tempfilename
'RH2KSQ03.wmf';tempfile-properties "XPR";}}
\end{equation*}
\end{itemize}

\pagebreak 

\item Una espira de 20 vueltas, radio $R_{1}=7,0$ [cm] y resistencia $R=$ $45
$ [$\Omega $] se ubica en el centro de un solenoide muy largo, que tiene $%
n=200$ [vueltas/metro], radio $R_{2}=5,0$ [cm] y es conc\'{e}ntrica a este 
\'{u}ltimo, como muestra la Fig. 2. Si por el solenoide circula una
corriente variable $I(t)=3\sqrt{t}$ [A], determine,

\begin{itemize}
\item El flujo magn\'{e}tico a trav\'{e}s de la espira.

\item La fem inducida $\varepsilon $ en la espira.

\item La magnitud y direcci\'{o}n de la corriente inducida en la espira $I$%
\begin{equation*}
\FRAME{itbpFU}{2.2693in}{2.9499in}{0in}{\Qcb{Figura 2}}{}{Figure}{\special%
{language "Scientific Word";type "GRAPHIC";maintain-aspect-ratio
TRUE;display "USEDEF";valid_file "T";width 2.2693in;height 2.9499in;depth
0in;original-width 2.2295in;original-height 2.9066in;cropleft "0";croptop
"1";cropright "1";cropbottom "0";tempfilename
'RH2LI804.wmf';tempfile-properties "XPR";}}
\end{equation*}
\end{itemize}

\item Problema 7, Gu\'{\i}a 9.
\end{enumerate}

\end{document}
