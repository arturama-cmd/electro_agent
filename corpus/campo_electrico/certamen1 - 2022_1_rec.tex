% Sections won't have numbers
% Make the page area as large as possible
% Set the first level numbering to arabic and the second level
% to alphabetic. The remaining two levels are arabic
%Change to \pagestyle{empty} to suppress numbers on following pages
%\pagestyle{empty}
% Do not reset outermost enumerate counter


\documentclass{article}
%%%%%%%%%%%%%%%%%%%%%%%%%%%%%%%%%%%%%%%%%%%%%%%%%%%%%%%%%%%%%%%%%%%%%%%%%%%%%%%%%%%%%%%%%%%%%%%%%%%%%%%%%%%%%%%%%%%%%%%%%%%%%%%%%%%%%%%%%%%%%%%%%%%%%%%%%%%%%%%%%%%%%%%%%%%%%%%%%%%%%%%%%%%%%%%%%%%%%%%%%%%%%%%%%%%%%%%%%%%%%%%%%%%%%%%%%%%%%%%%%%%%%%%%%%%%
\usepackage{amsmath}
\usepackage{sw20exm2}

\setcounter{MaxMatrixCols}{10}
%TCIDATA{OutputFilter=LATEX.DLL}
%TCIDATA{Version=5.50.0.2953}
%TCIDATA{<META NAME="SaveForMode" CONTENT="1">}
%TCIDATA{BibliographyScheme=Manual}
%TCIDATA{Created=Tuesday, October 02, 2012 12:12:00}
%TCIDATA{LastRevised=Thursday, May 12, 2022 10:21:14}
%TCIDATA{<META NAME="ViewSettings" CONTENT="0">}
%TCIDATA{<META NAME="GraphicsSave" CONTENT="32">}
%TCIDATA{<META NAME="DocumentShell" CONTENT="Exams and Syllabi\SW\SW Exam #1 - 8.5x14 Page">}
%TCIDATA{CSTFile=Exam.cst}

\setlength{\topmargin}{-0.75in}
\setlength{\textheight}{12.25in}
\setlength{\oddsidemargin}{0.0in}
\setlength{\evensidemargin}{0.0in}
\setlength{\textwidth}{6.5in}
\def\labelenumi{\arabic{enumi}.}
\def\theenumi{\arabic{enumi}}
\def\labelenumii{(\alph{enumii})}
\def\theenumii{\alph{enumii}}
\def\p@enumii{\theenumi.}
\def\labelenumiii{\arabic{enumiii}.}
\def\theenumiii{\arabic{enumiii}}
\def\p@enumiii{(\theenumi)(\theenumii)}
\def\labelenumiv{\arabic{enumiv}.}
\def\theenumiv{\arabic{enumiv}}
\def\p@enumiv{\p@enumiii.\theenumiii}
\pagestyle{plain}
\setcounter{secnumdepth}{0}
\newtheorem{theorem}{Theorem}
\newtheorem{acknowledgement}[theorem]{Acknowledgement}
\newtheorem{algorithm}[theorem]{Algorithm}
\newtheorem{axiom}[theorem]{Axiom}
\newtheorem{case}[theorem]{Case}
\newtheorem{claim}[theorem]{Claim}
\newtheorem{conclusion}[theorem]{Conclusion}
\newtheorem{condition}[theorem]{Condition}
\newtheorem{conjecture}[theorem]{Conjecture}
\newtheorem{corollary}[theorem]{Corollary}
\newtheorem{criterion}[theorem]{Criterion}
\newtheorem{definition}[theorem]{Definition}
\newtheorem{example}[theorem]{Example}
\newtheorem{exercise}[theorem]{Exercise}
\newtheorem{lemma}[theorem]{Lemma}
\newtheorem{notation}[theorem]{Notation}
\newtheorem{problem}[theorem]{Problem}
\newtheorem{proposition}[theorem]{Proposition}
\newtheorem{remark}[theorem]{Remark}
\newtheorem{solution}[theorem]{Solution}
\newtheorem{summary}[theorem]{Summary}
\newenvironment{proof}[1][Proof]{\noindent\textbf{#1.} }{\ \rule{0.5em}{0.5em}}
\input{tcilatex}
\begin{document}


\ 
\begin{equation*}
\FRAME{itbpF}{1.1303in}{0.8821in}{0in}{}{}{Figure}{\special{language
"Scientific Word";type "GRAPHIC";maintain-aspect-ratio TRUE;display
"USEDEF";valid_file "T";width 1.1303in;height 0.8821in;depth
0in;original-width 2.3531in;original-height 1.8291in;cropleft "0";croptop
"1";cropright "1";cropbottom "0";tempfilename
'RBRWME00.bmp';tempfile-properties "XPR";}}
\end{equation*}%
\ \ \ \ \ \ \ \ \ \ \ \ \ \ \ \ \ \ \ \ \ \ \ \ \ \ \ \ \ \ \ \ \ \ \ \ \ \
\ \ \ \ \ \ \ \ \ \ \textbf{Certamen 1 - Electromagnetismo}

\ \ \ \ \ \ \ \ \ \ \ \ \ \ \ \ \ \ \ \ \ \ \ \ \ \ \ \ \ \ \ \ \ \ \ \ \ \
\ \ \ \ \ \ \ \ \ \ \ \ \ \ \ \ \ \ \ \ \ \ 12/05/2022

\noindent\ \ \ \ \ \ \ \ \ \ \ \ \ \ \ \ \ \ \ \ \ \ \ \ \ \ \ \ \ \ \ \ \ \
\ \ \ \ \ \ \ \ \ \ \ \ \ \ \ Profesor: Arturo Fern\'{a}ndez P\'{e}rez 
\hspace*{\stretch{0.7500}}

\bigskip

\textbf{Instrucciones: }

\begin{itemize}
\item \emph{Responda de manera clara, ordenada y sin borrones. }

\item \emph{No se revisar\'{a}n respuestas poco claras o ilegibles. }

\item \emph{Puntaje Total: 59 puntos. Puntaje m\'{\i}nimo para la aprobaci%
\'{o}n: 30 puntos.}
\end{itemize}

\bigskip

\begin{enumerate}
\item Tres cargas puntuales $q_{1}=1,5$ $[nC]$, $q_{2}=3,5$ $[nC]$ y $%
q_{3}=-1,8$ $[nC]$\ se ubican en un sistema de referencia, cuyas coordenadas
(en cent\'{\i}metros, [cm]) se muestran en la Fig. 1. Determine:

\begin{itemize}
\item (16 puntos) La fuerza el\'{e}ctrica neta $\vec{F}$ ejercida sobre la
carga $q_{2}.$

\item (14 puntos) El campo el\'{e}ctrico neto $\vec{E}$ en el punto (1,2)
[cm]%
\begin{equation*}
\FRAME{itbpFU}{2.9758in}{2.4751in}{0in}{\Qcb{Figura 1}}{}{Figure}{\special%
{language "Scientific Word";type "GRAPHIC";maintain-aspect-ratio
TRUE;display "USEDEF";valid_file "T";width 2.9758in;height 2.4751in;depth
0in;original-width 7.0206in;original-height 5.8332in;cropleft "0";croptop
"1";cropright "1";cropbottom "0";tempfilename
'RBRWW304.wmf';tempfile-properties "XPR";}}
\end{equation*}%
\pagebreak 
\end{itemize}

\item Una barra con densidad lineal de carga uniforme $\lambda =3,6$ [$nC/m$%
] se ubica sobre el eje $x$, como muestra la Fig. 2, entre $x=5$ [cm] e $x=18
$ [cm\.{]}. Adem\'{a}s, en el punto (-2,6) [cm] se ubica una carga puntual $%
q=-16$ [$nC$]. Determine:

\begin{itemize}
\item (22 puntos) El campo el\'{e}ctrico $\vec{E}$ inducido por la barra en
el punto (-2,6) [cm].

\item (7 puntos) La fuerza el\'{e}ctrica $\vec{F}$ sobre la carga $q.$%
\begin{equation*}
\FRAME{itbpFU}{4.2073in}{2.9533in}{0in}{\Qcb{Figura 2}}{}{Figure}{\special%
{language "Scientific Word";type "GRAPHIC";maintain-aspect-ratio
TRUE;display "USEDEF";valid_file "T";width 4.2073in;height 2.9533in;depth
0in;original-width 6.9272in;original-height 4.8542in;cropleft "0";croptop
"1";cropright "1";cropbottom "0";tempfilename
'RBRWYK05.wmf';tempfile-properties "XPR";}}
\end{equation*}%
\begin{equation*}
\FRAME{itbpF}{4.2973in}{2.3047in}{0in}{}{}{Figure}{\special{language
"Scientific Word";type "GRAPHIC";maintain-aspect-ratio TRUE;display
"USEDEF";valid_file "T";width 4.2973in;height 2.3047in;depth
0in;original-width 8.2321in;original-height 4.4028in;cropleft "0";croptop
"1";cropright "1";cropbottom "0";tempfilename
'RBRWME03.bmp';tempfile-properties "XPR";}}
\end{equation*}
\end{itemize}
\end{enumerate}

\end{document}
